\documentclass[a4paper,11pt,twoside,openright]{report}


% ********* PACCHETTI *********
\usepackage[T1]{fontenc}	% codifica dei font in uscita
\usepackage[utf8]{inputenc} % codifica dei font in uscita
\usepackage[italian]{babel} % lingua del documento
\usepackage{graphicx}		% inserimento di immagini
\graphicspath{{images/}{../images/}}

\usepackage{fancyhdr}		% personalizzare lo stile della pagina
\usepackage{subfiles} % permette l'uso di più file tex
\usepackage{lipsum}			% testo casuale di riempimento
\usepackage{listings}		% per inserire codice
\usepackage[colorlinks=true,%
			linkcolor=black,
            urlcolor=blue,
            citecolor=black]{hyperref}		% uso di link interni al documento
\usepackage{color}			% uso di colori base
\usepackage{xcolor}			% uso di colori avanzati
\usepackage[many]{tcolorbox}		% creare riquadri personalizzati
\usepackage[nonumberlist,toc]{glossaries}		% uso del glossario
\usepackage[nottoc]{tocbibind}	% aggiungere la bibliografia alla TOC
\usepackage{emptypage}
%\usepackage{biblatex}
%\usepackage{natbib}

% ********* IMPOSTAZIONI *********

% profondita della TOC
\setcounter{secnumdepth}{3}

% definire C#
\newcommand{\Csharp}{%
  {\settoheight{\dimen0}{C}C\kern-.05em \resizebox{!}{\dimen0}{\raisebox{\depth}{\#}}}}
  
% definire batch
\lstdefinestyle{batch}{
    language={[WinXP]command.com},
    commentstyle=\color{bat-comm},
    keywordstyle=\color{bat-kw},
    keywords=[2]{@},
    keywords=[3]{\%1,\%~2,\%3,\%4, \%params\%, \%url\%, \%redirect\%},
    keywords=[4]{curl},
    keywordstyle=[2]\color{bat-at},
    keywordstyle=[3]\color{bat-str},
    keywordstyle=[4]\color{bat-cmd} 
}

% definire Javascript
\lstdefinelanguage{JavaScript}{
  keywords={typeof, new, true, false, catch, function, return, null, catch, switch, var, if, in, while, do, else, case, break},
  keywordstyle=\color{blue}\bfseries,
  ndkeywords={class, export, boolean, throw, implements, import, this},
  ndkeywordstyle=\color{darkgray}\bfseries,
  identifierstyle=\color{black},
  sensitive=false,
  comment=[l]{//},
  morecomment=[s]{/*}{*/},
  commentstyle=\color{purple}\ttfamily,
  stringstyle=\color{red}\ttfamily,
  morestring=[b]',
  morestring=[b]"
}


\pagestyle{headings}
% ************ END ***************

\makeglossaries				% alla compilazione crea il file .glo
%\documentclass[../RelazioneFinale.tex]{subfiles}

\newglossaryentry{Solution}
{
	name=Solution,
	description={File di configurazione dell'ambiente di sviluppo Visual studio che rappresenta un insieme di progetti}
}

\newglossaryentry{Macro componenti}
{
	name=Macro componente,
	description={Insieme di componenti (classi), può coincidere con un namespace (\Csharp) o package (Java)}
}

\newglossaryentry{Prodotto software}
{
	name=Prodotto software,
	description={Software sviluppati dall'azienda e coinvolti nella distribuzione del sistema Updater}
}

\newglossaryentry{Framework}
{
	name=Framework,
	description={Architettura logica di supporto su cui un software può essere progettato e realizzato facilitandone lo sviluppo da parte del programmatore}
}

\newglossaryentry{Code Behind}
{
	name=Code Behind,
	description={Termine utilizzato per descrivere il codice che viene unito agli oggetti definiti dal markup quando viene compilata una pagina XAML}
}

\newglossaryentry{Metro Design}
{
	name=Metro Design,
	description={Nuovo linguaggio di design sviluppato da Microsoft per i propri prodotti per la creazione di interfacce grafiche eleganti, veloci e moderne}
}

\newglossaryentry{TaskBar}
{
	name=TaskBar,
	description={Speciale barra degli strumenti dei sistemi operativi Windows in cui è possibile interagire con programmi del sistema o programmi eseguiti in background}
}

\newglossaryentry{Baloon}
{
	name=Baloon,
	description={Piccola finestra pop-up che informa l'utente dell'avvenimento di un problema non critico}
}

\newglossaryentry{Prerequisiti}
{
	name=Prerequisiti,
	description={Prodotti software sviluppati da terzi su cui si basa lo sviluppo e il funzionamento di altri prodotti software (per es. l'ambiente .NET o JDK)}
}

\newglossaryentry{HyperText Transfer Protocol}
{
	name=HyperText Transfer Protocol (HTTP),
	description={L'Hypertext Transfer Protocol (HTTP) è un protocollo a livello applicativo per sistemi informativi distribuiti}
}

\newglossaryentry{REpresentational State Transfer}
{
	name=REpresentational State Transfer (REST),
	description={Tipo di architettura software per i sistemi distribuiti nel web}
}


\newglossaryentry{Refactoring}
{
	name=Refactoring,
	description={Il refactoring è il processo di modifica di un sistema software che ha l'obiettivo di migliorare la struttura interna di esso mantenendo invariato il suo comportamento esterno}
}

\newglossaryentry{Data binding}
{
	name=Data binding,
	description={Tecnica che lega dei dati di un elemento fornitore (per es. una struttura dati) ad un elemento consumatore (per es. un componente grafico che rappresenta una lista) garantendo che entrambi siano sempre sincronizzati}
}

\newglossaryentry{Continuos Delivery}
{
	name=Continuos Delivery (CD),
	description={Continuous Delivery è una disciplina dello sviluppo software in cui si costruisce software garantendo che questo possa essere rilasciato in produzione in ogni momento}
}

\newglossaryentry{Continuos Integration}
{
	name=Continuos Integration (CI),
	description={La Continuos Integration (Integrazione continua) è una pratica dello sviluppo software in cui i membri di un team integrano il loro lavoro frequentemente con quello fatto dagli altri, spesso ogni individuo integra almeno una volta al giorno. Ogni integrazione è verificata da un sistema di build automatico che include test il quale avvisa tempestivamente se ci sono errori di integrazione. Tutto ciò porta a sviluppare un software coeso più rapidamente\cite{duvall2007continuous}}
}

\newglossaryentry{Design Pattern}
{
	name=Design Pattern,
	description={Soluzione progettuale generale ad un problema ricorrente}
}

\newglossaryentry{Repository}
{
	name=Repository,
	description={ambiente di un sistema informativo in cui vengono
gestiti metadati attraverso tabelle relazionali}
}		% inlude nel documneto tutti i termini del glossario

\begin{document}

	% !TEX encoding = UTF-8
% !TEX TS-program = pdflatex
% !TEX root = ../tesi.tex
% !TEX spellcheck = it-IT

\newcommand{\myName}{Eduard Bicego}                                    % autore
\newcommand{\myTitle}{Sviluppo di un sistema completo di update e installazione per prodotti CAD/CAM}                    
\newcommand{\myDegree}{Tesi di laurea triennale}                % tipo di tesi
\newcommand{\myUni}{Università degli Studi di Padova}           % università
\newcommand{\myFaculty}{Corso di Laurea in Informatica}         % facoltà
\newcommand{\myDepartment}{Dipartimento di Matematica}          % dipartimento
\newcommand{\myProf}{Massimo Marchiori}                                % relatore
\newcommand{\myLocation}{Padova}                                % dove
\newcommand{\myAA}{2015-2016}                                   % anno accademico
\newcommand{\myTime}{Settembre 2016}                                  % quando


%**************************************************************
% Frontespizio 
%**************************************************************
\begin{titlepage}

\begin{center}

\begin{LARGE}
\textbf{\myUni}\\
\end{LARGE}

\vspace{10pt}

\begin{Large}
\textsc{\myDepartment}\\
\end{Large}

\vspace{10pt}

\begin{large}
\textsc{\myFaculty}\\
\end{large}

\vspace{10pt}
\begin{figure}[htbp]
\begin{center}
\includegraphics[height=5cm]{logo-unipd}
\end{center}
\end{figure}
\vspace{25pt} 

\begin{LARGE}
\begin{center}
\textbf{\myTitle}\\
\end{center}
\end{LARGE}

\vspace{10pt} 

\begin{LARGE}
\textsl{\myDegree}\\
\end{LARGE}

\vfill

\begin{Large}
\textit{Relatore} \hfill \textit{Laureando}\\ 
\vspace{5pt} 
Prof. \myProf \hfill \myName 
\end{Large}

\vspace{30pt}

\line(1, 0){338} \\
\begin{large}
\textsc{Anno Accademico \myAA}
\end{large}

\end{center}
\end{titlepage} 
	\clearpage
	\input{layout/colophon}
	\clearpage
	%\cleardoublepage
\phantomsection
\thispagestyle{empty}
%\pdfbookmark{Dedica}{Dedica}

\null
\vspace {\stretch{0.5}}
\begin{flushright}
	\LARGE
	\emph{A Elena}
\end{flushright}
\vspace {\stretch{2}}
\null


	\setcounter{page}{1}
	\pagenumbering{roman}
	
	\cleardoublepage
\phantomsection
\pdfbookmark{Sommario}{Sommario}
\begingroup
\let\clearpage\relax
\let\cleardoublepage\relax
\let\cleardoublepage\relax

\chapter*{Sommario}

Il presente documento descrive il progetto sviluppato durante il periodo di stage del laureando Eduard Bicego presso l'azienda Salvagnini S.p.A. sotto supervisione del tutor aziendale Alberto Conz. 
		
		\noindent Il progetto richiedeva lo sviluppo di un sistema semplificato per la distribuzione ai clienti dei prodotti software sviluppati all'interno dell'azienda. Gli obiettivi erano quindi lo sviluppo di un'applicazione installabile lato client e un servizio REpresentational State Transfer (REST).

\endgroup			

\vfill

	% !TEX encoding = UTF-8
% !TEX TS-program = pdflatex
% !TEX root = ../tesi.tex
% !TEX spellcheck = it-IT

%**************************************************************
% Ringraziamenti
%**************************************************************
\cleardoublepage
\phantomsection
\pdfbookmark{Ringraziamenti}{ringraziamenti}

\begingroup
\let\clearpage\relax
\let\cleardoublepage\relax
\let\cleardoublepage\relax

\chapter*{Ringraziamenti}

\noindent \textit{Desidero ringraziare tutti coloro che in questo percorso universitario hanno condiviso una parte del cammino verso il mondo informatico insieme a me. In particolare ringrazio il Prof. Massimo Marchiori, relatore della mia tesi, per la sua immensa passione nell'insegnamento e il gruppo Leaf: Andrea, Davide, Cristian, Federico, Marco e Oscar per la fatica e la gioia condivisa quest'anno.}\\

\noindent \textit{Ringrazio con affetto i miei genitori: Gianpaolo e Rossella per l'enorme sostegno e l'immenso aiuto nel conseguire questo obiettivo.}\\

\noindent \textit{Ringrazio i miei amici e la stupenda famiglia dell'A\textsuperscript{2}CMMS che in questi anni ha consentito la mia crescita personale in un vortice di intense esperienze ed emozioni.}\\

\noindent \textit{Ringrazio poi il mio tutor aziendale Alberto Conz e il suo collega Davide Fasolo per i numerosi insegnamenti e spunti ricevuti durante tutto il periodo di stage.}\\

\noindent \textit{Infine desidero ringraziare chiunque, anche se solo per un momento, mi ha accompagnato fin qui e chi mi accompagnerà nel prosieguo.}\\
\bigskip

\noindent\textit{\myTime}
\hfill \myName

\endgroup


	
	\tableofcontents
	\addcontentsline{toc}{chapter}{Indice}
	
	\listoffigures
	
	%\listoftables
	

	\cleardoublepage
	\pagenumbering{arabic}

	\subfile{sections/Introduzione}
	\subfile{sections/Requisiti}
	\subfile{sections/Tecnologie}
	\subfile{sections/Architettura}
	\subfile{sections/Dettagli}
	\subfile{sections/Conclusioni}
	
	% --- BIBLIOGRAFIA ---
	\nocite{*}		
	\bibliographystyle{plain}
	\bibliography{Bibliografia}	
	
	% --- GLOSSARIO ---
	\glsaddall
	\printglossary
	
	\label{LastPage}
	

\end{document}