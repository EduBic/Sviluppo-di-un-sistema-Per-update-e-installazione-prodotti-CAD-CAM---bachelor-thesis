\section{Updater Client Library - Documentazione}
		\label{sec:UpdaterClientLibrary}
		
			\subsection{Introduzione}
			
			
			\subsection{Componenti}
			
				\subsubsection{Indice componenti}
				
				\begin{tabular}{lc}
					nome classe & pagina \\
					MyClass & \pageref{MyClass} \\
					\verb|UpdaterService| & \pageref{} \\
					\verb|ProductWrapper| & \pageref{} \\
					\verb|PrerequisiteWrapper| & \pageref{} \\
					\verb|AutomaticUpdatesController| & \pageref{} \\
					\verb|AutomaticUpdatesInstaller| & \pageref{} \\
					\verb|AutomaticUpdatesCheckerAndDownloader| & \pageref{} \\
					\verb|AppRestarter| & \pageref{} \\
					\verb|Models::ObjectManager| & \pageref{} \\
				
				\end{tabular}
				
				\subsubsection{ProductWrapper}

					\begin{figure}
						%\includegraphics[scale=•]{•}
					\end{figure}									
				
					\paragraph{Responsabilità}
					
					\paragraph{Descrizione metodi} \ \par
					
						%\hspace{4cm}
						
						\begin{lstlisting}
 main(stringa : string, secondo : Tipo) : void
						\end{lstlisting}
						Bla blas bla 
						
						\begin{lstlisting}
 void main2(VarArgs...args)
						\end{lstlisting}
				
					
				\subsubsection{PrerequisiteWrapper}
				
					%\begin{figure}
						%\includegraphics[scale=•]{•}
					%\end{figure}								
				
					\paragraph{Responsabilità}
					
					\paragraph{Descrizione metodi} \ \par
					
						
						\begin{lstlisting}
 main(stringa : string, secondo : Tipo) : void
						\end{lstlisting}
						Bla blas bla 
						
						\begin{lstlisting}
 void main2(VarArgs...args)
						\end{lstlisting}
	
		\paragraph*{}
		\label{MyClass}
		\begin{tcolorbox}[fonttitle=\bfseries, 
								adjusted title={\Large MyClass}, 
								breakable, 
								sharp corners=south,
								colback=white, 
								colframe=white!60!black]
								
				\begin{description}%[leftmargin=0.7cm,labelwidth=!]
				
					\item[Responsabilità] \ \par 
        				asnfogandgonadonfkln fjsdnf kans njsdf nkajsnf 
        				
        			\tcbline 
        				
        			\item[Descrizione metodi] \ \par
        				
        				\begin{lstlisting}
 main(stringa : string, secondo : Tipo) : void
						\end{lstlisting}
						Bla blas bla 
						
						\begin{lstlisting}
 void main2(VarArgs...args)
						\end{lstlisting}
        				
				\end{description}  
				
			\end{tcolorbox}	
	
		\paragraph*{}
		\label{AppRestarter}
		\begin{tcolorbox}[fonttitle=\bfseries, 
								adjusted title={\Large AppRestarter}, 
								breakable, 
								sharp corners=south,
								colback=white, 
								colframe=white!60!black]
								
				\begin{description}%[leftmargin=0.7cm,labelwidth=!]
				
					\item[Responsabilità] \ \par 
        				Gestire la funzionalità di riavvio dell'applicazione come amministratore per le operazioni:
        				\begin{itemize}
        					\item Aggiornamenti;
        					\item Installazioni;
        					\item Operazioni automatiche in background.
        				\end{itemize}
        			\tcbline 
        				
        			\item[Descrizione metodi] \ \par
        				
        				\begin{lstlisting}
public AppRestarter()
						\end{lstlisting}
						Costruttore della classe.
						
						\begin{lstlisting}
bool IsAppRunningAsAdministrator()
						\end{lstlisting}
						Metodo per conoscere se l'applicazione è già in esecuzione con i permessi d'amministratore.
						
						\begin{lstlisting}
void RestartAppAsAdminForUpdate(long idProduct)
						\end{lstlisting}
						Metodo per riavviare l'applicazione con i permessi d'amministratore gestendo gli aggiornamenti del prodotto software indicato nel parametro \verb|idProduct|.  
						
						\begin{lstlisting}
void RestartAppAsAdminForInstall(long idProduct)
						\end{lstlisting}
						Metodo per riavviare l'applicazione con i permessi d'amministratore gestendo l'installazione del prodotto software indicato nel parametro \verb|idProduct|.  	
						
						\begin{lstlisting}
void RestartAppAsAdminForBackgroundOperation
			    (long idProduct)
						\end{lstlisting}
						Metodo per riavviare l'applicazione con i permessi d'amministratore gestendo operazioni in background (download, installazione e aggioramento) del prodotto software indicato nel parametro \verb|idProduct|.  					
        				
				\end{description}  
				
			\end{tcolorbox}	