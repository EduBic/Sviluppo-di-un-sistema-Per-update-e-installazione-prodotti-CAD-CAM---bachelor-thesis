% !TEX encoding = UTF-8
% !TEX TS-program = pdflatex
% !TEX root = ../tesi.tex
% !TEX spellcheck = it-IT

\newcommand{\myName}{Eduard Bicego}                                    % autore
\newcommand{\myTitle}{Sviluppo di un sistema completo di update e installazione per prodotti CAD/CAM}                    
\newcommand{\myDegree}{Tesi di laurea triennale}                % tipo di tesi
\newcommand{\myUni}{Università degli Studi di Padova}           % università
\newcommand{\myFaculty}{Corso di Laurea in Informatica}         % facoltà
\newcommand{\myDepartment}{Dipartimento di Matematica}          % dipartimento
\newcommand{\myProf}{Massimo Marchiori}                                % relatore
\newcommand{\myLocation}{Padova}                                % dove
\newcommand{\myAA}{2015-2016}                                   % anno accademico
\newcommand{\myTime}{Settembre 2016}                                  % quando


%**************************************************************
% Frontespizio 
%**************************************************************
\begin{titlepage}

\begin{center}

\begin{LARGE}
\textbf{\myUni}\\
\end{LARGE}

\vspace{10pt}

\begin{Large}
\textsc{\myDepartment}\\
\end{Large}

\vspace{10pt}

\begin{large}
\textsc{\myFaculty}\\
\end{large}

\vspace{10pt}
\begin{figure}[htbp]
\begin{center}
\includegraphics[height=5cm]{logo-unipd}
\end{center}
\end{figure}
\vspace{25pt} 

\begin{LARGE}
\begin{center}
\textbf{\myTitle}\\
\end{center}
\end{LARGE}

\vspace{10pt} 

\begin{LARGE}
\textsl{\myDegree}\\
\end{LARGE}

\vfill

\begin{Large}
\textit{Relatore} \hfill \textit{Laureando}\\ 
\vspace{5pt} 
Prof. \myProf \hfill \myName 
\end{Large}

\vspace{30pt}

\line(1, 0){338} \\
\begin{large}
\textsc{Anno Accademico \myAA}
\end{large}

\end{center}
\end{titlepage} 