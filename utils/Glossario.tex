%\documentclass[../RelazioneFinale.tex]{subfiles}

\newglossaryentry{Solution}
{
	name=Solution,
	description={File di configurazione dell'ambiente di sviluppo Visual studio che rappresenta un insieme di progetti}
}

\newglossaryentry{Macro componenti}
{
	name=Macro componente,
	description={Insieme di componenti (classi), può coincidere con un namespace (\Csharp) o package (Java)}
}

\newglossaryentry{Prodotto software}
{
	name=Prodotto software,
	description={Software sviluppati dall'azienda e coinvolti nella distribuzione del sistema Updater}
}

\newglossaryentry{Framework}
{
	name=Framework,
	description={Architettura logica di supporto su cui un software può essere progettato e realizzato facilitandone lo sviluppo da parte del programmatore}
}

\newglossaryentry{Code Behind}
{
	name=Code Behind,
	description={Termine utilizzato per descrivere il codice che viene unito agli oggetti definiti dal markup quando viene compilata una pagina XAML}
}

\newglossaryentry{Metro Design}
{
	name=Metro Design,
	description={Nuovo linguaggio di design sviluppato da Microsoft per i propri prodotti per la creazione di interfacce grafiche eleganti, veloci e moderne}
}

\newglossaryentry{TaskBar}
{
	name=TaskBar,
	description={Speciale barra degli strumenti dei sistemi operativi Windows in cui è possibile interagire con programmi del sistema o programmi eseguiti in background}
}

\newglossaryentry{Baloon}
{
	name=Baloon,
	description={Piccola finestra pop-up che informa l'utente dell'avvenimento di un problema non critico}
}

\newglossaryentry{Prerequisiti}
{
	name=Prerequisiti,
	description={Prodotti software sviluppati da terzi su cui si basa lo sviluppo e il funzionamento di altri prodotti software (per es. l'ambiente .NET o JDK)}
}

\newglossaryentry{HyperText Transfer Protocol}
{
	name=HyperText Transfer Protocol (HTTP),
	description={L'Hypertext Transfer Protocol (HTTP) è un protocollo a livello applicativo per sistemi informativi distribuiti}
}

\newglossaryentry{REpresentational State Transfer}
{
	name=REpresentational State Transfer (REST),
	description={Tipo di architettura software per i sistemi distribuiti nel web}
}


\newglossaryentry{Refactoring}
{
	name=Refactoring,
	description={Il refactoring è il processo di modifica di un sistema software che ha l'obiettivo di migliorare la struttura interna di esso mantenendo invariato il suo comportamento esterno}
}

\newglossaryentry{Data binding}
{
	name=Data binding,
	description={Tecnica che lega dei dati di un elemento fornitore (per es. una struttura dati) ad un elemento consumatore (per es. un componente grafico che rappresenta una lista) garantendo che entrambi siano sempre sincronizzati}
}

\newglossaryentry{Continuos Delivery}
{
	name=Continuos Delivery (CD),
	description={Continuous Delivery è una disciplina dello sviluppo software in cui si costruisce software garantendo che questo possa essere rilasciato in produzione in ogni momento}
}

\newglossaryentry{Continuos Integration}
{
	name=Continuos Integration (CI),
	description={La Continuos Integration (Integrazione continua) è una pratica dello sviluppo software in cui i membri di un team integrano il loro lavoro frequentemente con quello fatto dagli altri, spesso ogni individuo integra almeno una volta al giorno. Ogni integrazione è verificata da un sistema di build automatico che include test il quale avvisa tempestivamente se ci sono errori di integrazione. Tutto ciò porta a sviluppare un software coeso più rapidamente\cite{duvall2007continuous}}
}

\newglossaryentry{Design Pattern}
{
	name=Design Pattern,
	description={Soluzione progettuale generale ad un problema ricorrente}
}

\newglossaryentry{Repository}
{
	name=Repository,
	description={ambiente di un sistema informativo in cui vengono
gestiti metadati attraverso tabelle relazionali}
}